\documentclass[a4paper,14pt]{article}

%%% Работа с русским языком
\usepackage{cmap}					% поиск в PDF
\usepackage{mathtext} 				% русские буквы в формулах
\usepackage[T2A]{fontenc}			% кодировка
\usepackage[utf8]{inputenc}			% кодировка исходного текста
\usepackage[english,russian]{babel}	% локализация и переносы
\usepackage{indentfirst}
\frenchspacing

\newcommand{\bz}{\mathbf{z}}
\newcommand{\bx}{\mathbf{x}}
\newcommand{\by}{\mathbf{y}}
\newcommand{\bv}{\mathbf{v}}
\newcommand{\bw}{\mathbf{w}}
\newcommand{\ba}{\mathbf{a}}
\newcommand{\bb}{\mathbf{b}}
\newcommand{\bp}{\mathbf{p}}
\newcommand{\bq}{\mathbf{q}}
\newcommand{\bt}{\mathbf{t}}
\newcommand{\bu}{\mathbf{u}}
\newcommand{\bT}{\mathbf{T}}
\newcommand{\bX}{\mathbf{X}}
\newcommand{\bZ}{\mathbf{Z}}
\newcommand{\bS}{\mathbf{S}}
\newcommand{\bH}{\mathbf{H}}
\newcommand{\bW}{\mathbf{W}}
\newcommand{\bY}{\mathbf{Y}}
\newcommand{\bU}{\mathbf{U}}
\newcommand{\bQ}{\mathbf{Q}}
\newcommand{\bP}{\mathbf{P}}
\newcommand{\bA}{\mathbf{A}}
\newcommand{\bB}{\mathbf{B}}
\newcommand{\bC}{\mathbf{C}}
\newcommand{\bE}{\mathbf{E}}
\newcommand{\bF}{\mathbf{F}}
\newcommand{\bomega}{\boldsymbol{\omega}}
\newcommand{\btheta}{\boldsymbol{\theta}}
\newcommand{\bgamma}{\boldsymbol{\gamma}}
\newcommand{\bdelta}{\boldsymbol{\delta}}
\newcommand{\bPsi}{\boldsymbol{\Psi}}
\newcommand{\bpsi}{\boldsymbol{\psi}}
\newcommand{\bxi}{\boldsymbol{\xi}}
\newcommand{\bchi}{\boldsymbol{\chi}}
\newcommand{\bzeta}{\boldsymbol{\zeta}}
\newcommand{\blambda}{\boldsymbol{\lambda}}
\newcommand{\beps}{\boldsymbol{\varepsilon}}
\newcommand{\bZeta}{\boldsymbol{Z}}
% mathcal
\newcommand{\cX}{\mathcal{X}}
\newcommand{\cY}{\mathcal{Y}}
\newcommand{\cW}{\mathcal{W}}

\newcommand{\dH}{\mathbb{H}}
\newcommand{\dR}{\mathbb{R}}
\newcommand{\dE}{\mathbb{E}}
% transpose
\newcommand{\T}{^{\mathsf{T}}}

\renewcommand{\epsilon}{\ensuremath{\varepsilon}}
\renewcommand{\phi}{\ensuremath{\varphi}}
\renewcommand{\kappa}{\ensuremath{\varkappa}}
\renewcommand{\le}{\ensuremath{\leqslant}}
\renewcommand{\leq}{\ensuremath{\leqslant}}
\renewcommand{\ge}{\ensuremath{\geqslant}}
\renewcommand{\geq}{\ensuremath{\geqslant}}
\renewcommand{\emptyset}{\varnothing}

%%% Дополнительная работа с математикой
\usepackage{amsmath,amsfonts,amssymb,amsthm,mathtools} % AMS
\usepackage{icomma} % "Умная" запятая: $0,2$ --- число, $0, 2$ --- перечисление

%% Номера формул
%\mathtoolsset{showonlyrefs=true} % Показывать номера только у тех формул, на которые есть \eqref{} в тексте.
%\usepackage{leqno} % Нумереация формул слева

%% Свои команды
\DeclareMathOperator{\sgn}{\mathop{sgn}}

%% Перенос знаков в формулах (по Львовскому)
\newcommand*{\hm}[1]{#1\nobreak\discretionary{}
	{\hbox{$\mathsurround=0pt #1$}}{}}

%%% Работа с картинками
\usepackage{graphicx}  % Для вставки рисунков
\setlength\fboxsep{3pt} % Отступ рамки \fbox{} от рисунка
\setlength\fboxrule{1pt} % Толщина линий рамки \fbox{}
\usepackage{wrapfig} % Обтекание рисунков текстом

%%% Работа с таблицами
\usepackage{array,tabularx,tabulary,booktabs} % Дополнительная работа с таблицами
\usepackage{longtable}  % Длинные таблицы
\usepackage{multirow} % Слияние строк в таблице

%%% Теоремы
\theoremstyle{plain} % Это стиль по умолчанию, его можно не переопределять.
\newtheorem{theorem}{Теорема}[section]
\newtheorem{proposition}[theorem]{Утверждение}

\theoremstyle{definition} % "Определение"
\newtheorem{corollary}{Следствие}[theorem]
\newtheorem{problem}{Задача}[section]

\theoremstyle{remark} % "Примечание"
\newtheorem*{nonum}{Решение}

%%% Программирование
\usepackage{etoolbox} % логические операторы

%%% Страница
\usepackage{extsizes} % Возможность сделать 14-й шрифт
\usepackage{geometry} % Простой способ задавать поля
\geometry{top=25mm}
\geometry{bottom=35mm}
\geometry{left=35mm}
\geometry{right=20mm}
%
%\usepackage{fancyhdr} % Колонтитулы
% 	\pagestyle{fancy}
%\renewcommand{\headrulewidth}{0pt}  % Толщина линейки, отчеркивающей верхний колонтитул
% 	\lfoot{Нижний левый}
% 	\rfoot{Нижний правый}
% 	\rhead{Верхний правый}
% 	\chead{Верхний в центре}
% 	\lhead{Верхний левый}
%	\cfoot{Нижний в центре} % По умолчанию здесь номер страницы

\usepackage{setspace} % Интерлиньяж
%\onehalfspacing % Интерлиньяж 1.5
%\doublespacing % Интерлиньяж 2
%\singlespacing % Интерлиньяж 1

\usepackage{lastpage} % Узнать, сколько всего страниц в документе.

\usepackage{soul} % Модификаторы начертания

\usepackage{hyperref}
\usepackage[usenames,dvipsnames,svgnames,table,rgb]{xcolor}
\hypersetup{				% Гиперссылки
	unicode=true,           % русские буквы в раздела PDF
	pdftitle={Заголовок},   % Заголовок
	pdfauthor={Автор},      % Автор
	pdfsubject={Тема},      % Тема
	pdfcreator={Создатель}, % Создатель
	pdfproducer={Производитель}, % Производитель
	pdfkeywords={keyword1} {key2} {key3}, % Ключевые слова
	colorlinks=true,       	% false: ссылки в рамках; true: цветные ссылки
	linkcolor=red,          % внутренние ссылки
	citecolor=black,        % на библиографию
	filecolor=magenta,      % на файлы
	urlcolor=cyan           % на URL
}

\usepackage{csquotes} % Еще инструменты для ссылок

% \usepackage[style=authoryear,maxcitenames=2,backend=biber,sorting=nty]{biblatex}
% \addbibresource{references.bib}
\usepackage[numbers]{natbib}

\usepackage{multicol} % Несколько колонок

\usepackage{tikz} % Работа с графикой
\usepackage{pgfplots}
\usepackage{pgfplotstable}


\author{Daniil Kazachkov, Andrew Filatov}
\title{\textbf{Body-Lightning ID Diffusion}}
\date{\today}

\begin{document}
	\maketitle
	\section{Abstract}

		В данной работе мы представляем архитектуру \textit{BoLID} - решение для интеграции body identity, позволяющее достигать более точной персонализации по сравнению с методами, учитывающими лишь лицо.
		Основные изменения заключены в энкодере пользовательского изображения.

	$\mathbf{keywords:}$ Machine Learning, Diffusion Models.

	\section{Introduction}
		Большой вклад в область генерации изображений внесли такие модели, как U-Net[link], DALL-E 2 [link], Imagen [link], Stable Diffusion [link].
		Сначала пользователи смогли получать картинки по текстовому промту, что является сильным ограничением. В DALLE-2 [link] пошли дальше и сделали попытку встроить изображение в диффузионную модель.
		Дальнейшее развитие архитектур привело к созданию легковесных адаптеров, например IP-Adapter [link], разделяющих механизм cross-attention для текстовых
		признаков и признаков изображения. Это позволило ввести элементы контроля генерации, подобно ControlNet [link], и использовать other custom models fine-tuned from the same base model.
		В работе InstantID [link] и PuLID [link] авторы включают дополнительный контроль за facial и landmark identity (ID), что дает сильный прирост в ID fidelity.

		Однако каждая из этих работ концентрируется лишь на учете особенностей лица, когда как тело пользователя остается без внимания.
		В данной работе я устраняю этот недостаток, встраивая в процесс генерации bodyID пользователя. Важной задачей является выделение значимых деталей с референсных изображений.
		Как в InstantID restrict yourself	to five key face points for a more generalized constraint, так и мы выделяем 12 points для туловища, что позволяет учитывать пропорции и форму тела.
		Для улучшения ID similarity мы дообучаем модель на датасетах, сгруппированных по уникальным идентификаторам пользователя, где каждый ID представлен серией снимков из разных ракурсов.
		Это обеспечивает более целостную генерацию, основанную на единичной входной фотографии, сохраняя физические пропорции пользователя вместе с его чертами лица.

		Ключевые изменения касаются двух модулей:
		\begin{enumerate}
			\item \textbf{Body-ID Encoder}, обучаемый экстрагировать особенности тела (body landmarks), аналогично подходу InstantID~\cite{wang2024instantid}, но для 12 основных ключевых точек туловища.
			\item \textbf{Body-Condition Module}, действующий как адаптер, схожий с IP-Adapter~\cite{ye2023ip-adapter}, где \textit{cross-attention} разделяется между признаками текста, лица и тела пользователя.
		\end{enumerate}

		\subsection{Loss Functions and Training Objective}
		\begin{itemize}
			\item \textbf{Reconstruction Loss:} минимизирует расхождение между сгенерированным изображением и референсным образцом при наличии настраиваемого условия (bodyID).
			\item \textbf{Identity Consistency Loss:} аналогично InstantID~\cite{wang2024instantid}, но расширено до корпуса и основных точек тела, для обеспечения правильной пропорциональности.
			\item \textbf{CLIP-based Similarity Loss:} использует EVA-CLIP~\cite{sun2023evaclipimprovedtrainingtechniques} для обеспечения семантической близости к заданному тексту и общему стилю входного изображения.
		\end{itemize}

	\section{Related Works}
		[Будет заполнено позже]

	\section{Method}
		[будет заполнено позже]

	\section{Computational experiment}
	\subsection{Implementation Details}
		Мы строим нашу модель на основе SDXL \citep{podell2023sdxlimprovinglatentdiffusion} и (?) шагов SDXL-Lightning \citep{lin2024-sdxllightning}. Для ID encoder мы используем (?)
		как модель распознавания тела, и EVA-CLIP \citep{sun2023evaclipimprovedtrainingtechniques} как CLIP Image encoder. Наш датасет состоит из приблизительно 50 тысяч картинок
		высокого разрешения, собранных из интернета, с подписями, автоматически сгенерированными с помощью BLIP-2 \citep{pmlr-v202-li23q} суммарно. Наш процесс обучения состоит из (?) шагов.

		\textbf{[при получении финальной модели этот раздел будет дописан]}

	\subsection{Basic experiment description}
		Бейзлайном выступает модель

	\subsection{Dataset}
		Поскольку наша цель - встроить тело пользователя, важно дообучить модель на множестве ракурсов одного человека.
		Поэтому в качестве датасета были выбраны высококачественные (с разрешением не менее $1000\times 1000$ px) фотографии знаменитостей. Для каждого человека собирался пакет из
		не менее 100 фотографий с сайта \href{https://www.theplace.ru/photos/}{theplace} с помощью написанного \href{https://github.com/wolkendolf/2025-project-DiffModels/tree/main/parser}{парсера}.
		Датасет доступен по \href{https://drive.google.com/drive/folders/1gP83US8DSw-OM0Fc1MlOYFVd0QJ7oJmy?usp=sharing}{ссылке}.

	\subsection{Results}
		Ниже будет представлен макет таблицы с результатами эксперимента.
		\begin{figure}[h]
			\centering
			\includegraphics[width=0.7\textwidth]{images/test_fig_compare.png} % путь к файлу
			\label{fig:test_fig_compare}
		\end{figure}

		\begin{figure}[h]
			\centering
			\includegraphics[width=0.7\textwidth]{images/test_quantitve_compare.png} % путь к файлу
			\label{fig:test_quantitve_compare}
		\end{figure}
		\textbf{Body Sim.} представляет собой косинусное сходство ID, извлеченными с помощью (?). \textbf{CLIP-T} [link] измеряет способность следовать подсказкам.
		\textbf{CLIP-I} для количественной оценки сходства изображений CLIP между двумя изображениями до и после вставки идентификатора.
		Более высокая метрика CLIP-I указывает на меньшее изменение элементов изображения (таких как фон, композиция, стиль) после вставки идентификатора,
		что говорит о меньшей степени нарушения поведения исходной модели.

	\nocite{*}
		\bibliographystyle{unsrt} % Use a numbered bibliography style
		\bibliography{references.bib}
	% \printbibliography
\end{document}
